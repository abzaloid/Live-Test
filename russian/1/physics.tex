\documentclass[12pt]{article}
\usepackage[russian]{babel}
\usepackage[utf8]{inputenc}
\usepackage[margin=0.5in]{geometry}
\usepackage{amsmath}
\title{StudySpace}
\author{Физика}
\date{vk.com/studyspacekz}

\begin{document}

\maketitle

1. Единицы работы в системе СИ

A) Н
B) Н/м
C) м/Дж
D) м/$c^2$
E) Дж

2. Термодинамической системе передано количество теплоты 200 Дж. Если при этом она совершила работу 400 Дж, то ее внутренняя энергия

A) увеличилась на 200 Дж.
B) увеличилась на 400 Дж.
C) уменьшилась на 400 Дж.
D) осталась без изменения.
E) уменьшилась на 200 Дж.

3. Тепловая машина за один цикл получает от нагревателя количество теплоты 10 Дж и отдает холодильнику 6 Дж. КПД машины 

A) 68\%.
B) 40\%.
C) 67\%.
D) 38\%.
E) 60\%.

5. Физическая величина, являющаяся силовой характеристикой электростатического поля, - это

A) напряженность.
B) элементарный заряд.
C) потенциал.
D) диэлектрическая проницаемость среды.
E) заряд.

6. Электрический ток – это

A) упорядоченное движение нейтральных частиц.
B) беспорядочное движение электрических зарядов.
C) беспорядочное движение электрических зарядов и 
упорядоченное движение нейтральных частиц.
D) упорядоченное движение электрических зарядов.
E) беспорядочное движение нейтральных частиц. 

7. Если ток, протекающий через электролит, увеличить в 4 раза, то масса вещества, выделившаяся при электролизе за время t

A) увеличится в 2 раза.
B) увеличится в 16 раз.
C) увеличится в 4 раза.
D) увеличится в 8 раз.
E) не изменится. 

8. Закон электромагнитной индукции выражается формулой

A) $F=B \ell \upsilon \sin\alpha$
B) $A=I\triangle \Phi$
C) $\epsilon_i=-\triangle \Phi/\triangle t$
D) $\Phi=L\cdot I$
E) $W=LI^2/2$

9. Естественная радиоактивность – это

A) превращение ядер атомов при облучении $\alpha$-частицами.
B) вырывание электронов из вещества под действием света.
C) превращение ядер при бомбардировке $\beta$-частицами.
D) превращение ядер под влиянием $\gamma$-лучей.
E) самопроизвольное превращение ядер. 

10. Автомобиль при торможении до полной остановки прошел 200 м. Если начальная скорость его движения была 72 км/ч, то он двигался с ускорением

A) -1 м/$c^2$
B) -0,8 м/$c^2$
C) 2 м/$c^2$
D) 4 м/$c^2$
E) -2 м/$c^2$

11. Тело проходит путь 4,9 м при свободном падении из состояния покоя за время (g=9,8 м/$c^2$)

A) 10 с.
B) 1 с.
C) 0,1 с.
D) 0,2 с.
E) 2 с. 

12. Насос, двигатель которого развивает мощность 25кВт, поднимает 100$\text{м}^3$ нефти на высоту 6м за 8минут. 
КПД установки ($\rho_\text{нефти}$ = 800 кг/$\text{м}^3$; g = 10 м/$c^2$)

A) 30\%
B) 50\%
C) 40\%
D) 100\%
E) 60\%

13. Число молей идеального газа, находящегося в сосуде объемом V при концентрации n, равно

A) $\nu = \cfrac{nV}{N_A}$
B) $\nu = \cfrac{VN_A}{nR}$
C) $\nu = \cfrac{nV}{k}$
D) $\nu = \cfrac{nRV}{N_A}$
E) $\nu = \cfrac{nV}{R}$

14. Для того чтобы расплавить за 100 мин 6 кг свинца, взятого при температуре плавления, мощность нагревателя должна быть $\left( \lambda=22,6 \cfrac{\text{кДж}}{\text{кг}}\right)$

A) 0,226Вт
B) 226 Вт
C) 81300кВт
D) 22,6 Вт
E) 13500кВт

15. Для того, чтобы периоды колебаний тела массой 200г, подвешенного на нити длиной 1м (математический маятник) и этого же тела, подвешенного на пружине (пружинный маятник) были равны, жесткость пружины должна равняться (g=10м/$c^2$)

A) 0,5 Н/м.
B) $\sqrt{2}$  Н/м
C) 20 Н/м.
D) 5 Н/м.
E) 2 Н/м. 

16. Поплавок на волнах за 20 с совершил 30 колебаний, а на расстоянии 20 м наблюдатель насчитал 10 

гребней. Скорость волны равна

A) 2 м/с.

B) 1 м/с.

C) 5 м/с.

D) 4 м/с.

E) 3 м/с. 

17. Отраженный радиоимпульс возвратился на Землю через 2,56 с от начала его посылки, то расстояние от Земли до Луны (с =$3\cdot10^8$м/с)

A) 768 000 км.
B) 76 800 км.
C) 3 840 000 км.
D) 384 000 км. 
E) 3 840 км.

18. Поляризация звуковых волн в воздухе

A) Не существует, т.к. звуковые волны – поперечные волны.
B) Существует, т.к. звуковые волны – продольные волны.
C) Существует, т.к. звуковые волны – поперечные волны.
D) Не существует, т.к. звуковые волны могут интерферировать.
E) Не существует, т.к. звуковые волны – продольные волны. 
19. Большим импульсом обладают фотоны излучения

A) желтого.
B) красного.
C) оранжевого.
D) зеленого.
E) синего. 

20. Точка движется по окружности с постоянной скоростью 0,5 м/с. Вектор скорости изменяет направление на $\triangle\phi=30^o$ за 2 с. Нормальное ускорение точки

A) 0,13 м/$c^2$
B) 13 м/$c^2$
C) 1,3 м/$c^2$
D) 7,5 м/$c^2$
E) 1,5 м/$c^2$

21. При уменьшении расстояния между двумя телами на 60 м сила их взаимного притяжения увеличилась на 69\%. Первоначальное расстояние между телами

A) 130 м.
B) 260 м.
C) 690 м.
D) 360 м.
E) 630 м. 

22. Тело свободно падает из состояния покоя с высоты 80 м. Его перемещение в последнюю секунду падения

A) 5 м.
B) 15 м.
C) 25 м.
D) 45 м.
E) 35 м.

23. Тело массой 0,2 кг падает в вязкой среде с высоты 1 м с ускорением 8 м/$c^2$. Изменение импульса тела равно

A) 0,6 кг·м/с
B) 0,9 кг·м/с
C) 0,5 кг·м/с
D) 0,8 кг·м/с
E) 0,7 кг·м/с

24. Из ствола пушки вылетает снаряд под углом $60^o$ к горизонту со скоростью 800м/с относительно Земли. Если масса снаряда 10кг, а масса пушки 2 тонны, то скорость отдачи пушки

A) 3 м/с
B) 2,5 м/с
C) 0,8 м/с
D) 2 м/с
E) 1 м/с

25. При съемке автомобиля длиной 4 м, пленка фотоаппарата располагалась от объектива на расстоянии 6,0 см. Длина негативного изображения получилась 3,2 cм, расстояние с которого снимали автомобиль равняется 

A) 10 м
B) 6,5 м
C) 12 м
D) 8,5 м
E) 7,5 м

\end{document}