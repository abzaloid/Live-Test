\documentclass[12pt]{article}
\usepackage[russian]{babel}
\usepackage[utf8]{inputenc}
% \usepackage[margin=0.5in]{geometry}
\usepackage{amsmath}

\title{StudySpace}
\author{vk.com/studyspacekz}
\date{Октябрь, 2015 г.}
\begin{document}

\maketitle

1. Вычислите: $3\cfrac{1}{3}:1\cfrac{1}{9}$

A) $\cfrac{1}{3}$
B) $\cfrac{7}{9}$
C) $2\cfrac{1}{3}$
D) $\cfrac{2}{3}$
E) $3$

2. За 2 билета на стадион уплатили 300 тенге. За пять таких билетов уплатили

A) 850 тенге
B) 750 тенге
C) 450 тенге
D) 650 тенге
E) 1500 тенге

3. Прологарифмируйте выражение $25а^2$ по основанию 5, где $а > 0$

A) $2\log_5 a$
B) $5+5\log_5 a$
C) $2 + 2\log_5 a$
D) $5$
E) $5\log_5 a$


4. От станции C в направлении D отправился скорый поезд, проходящий в час 70 км, а через час от станции D в направлении к станции C вышел товарный поезд со скоростью 45 км/ч. На каком расстоянии от D встретились поезда, если длина перегона CD равна 530 км? 

A) 180 км.
B) 190 км.
C) 200 км.
D) 210 км.
E) 220 км.


5. Решите уравнение: $9^x-6=3^x$

A) 6
B) -2
C) 1
D) 3
E) 0

6. Решите неравенство: $(x - 6)(x + 4) \le 0$. 

A) $[4; 6]$. 
B) $[-6; 4]$. 
C) $(-\infty; -4) \cup (6; +\infty). 
D) (-\infty; -4] \cup [6; +\infty)$.
E) $[-4; 6]$.

7. Вычислить: $\sin\cfrac{8\pi}{7}\cos\cfrac{\pi}{7}-\sin\cfrac{\pi}{7}\cos\cfrac{8\pi}{7}$

A) $\cfrac{\pi}{7}$
B) $0$
C) $-1$
D) $-\cfrac{\pi}{7}$
E) $1$

8. Решите уравнение: $\cos^2x-\sin^2x=\cfrac{\sqrt{3}}{2}$

A) $\pm \cfrac{\pi}{6} + \pi n, \quad n \in \mathbf{Z}$
B) $\cfrac{\pi}{12} + \pi n, \quad n \in \mathbf{Z}$
C) $\pm \cfrac{\pi}{12} + \pi n, \quad n \in \mathbf{Z}$
D) $\pm \cfrac{\pi}{12} + 2 \pi n, \quad n \in \mathbf{Z}$
E) $\pm \cfrac{\pi}{6} + 2 \pi n, \quad n \in \mathbf{Z}$

9. Найдите множество значений функции $y=2\cos^2x+7$

A) $(0; 9)$
B) $[7; 9]$
C) $[-1; 1]$
D) $(-1; 6)$
E) $(-\pi; 0)$

10. Решите систему уравнений: \begin{cases} 
x+2y=5 \\
-x+7y=13 
\end{cases}

A) (2; 1,5)
B) (-1; 3)
C) (1; 2)
D) (3; 1)
E) (-5; 5)

11. Решите уравнение: $\log_2\log_3(\tg x)=1$

A) $x=\cfrac{\pi}{2} + 2\pi k, \quad k \in \mathbf{Z}$
B) $x=arctg 9 + \pi k, \quad k \in \mathbf{Z}$
C) $x=\cfrac{\pi}{6} + \pi k, \quad k \in \mathbf{Z}$
D) $x=\cfrac{\pi}{4} + \pi k, \quad k \in \mathbf{Z}$
E) $x=\cfrac{\pi}{3} + \pi k, \quad k \in \mathbf{Z}$

12. Решите уравнение: $\sqrt{10 + 3\sqrt{2x - 3}}=5$

A) $3\cfrac{1}{4}$
B) $0$
C) $-14$
D) $7$
E) $14$

13. Вычислите: $\cfrac{3\cos\alpha-5\sin\alpha}{2\cos\alpha-\sin\alpha}$, при $\tg\alpha=1$

A) $8$
B) $1$
C) $2$
D) $-1$
E) $-2$

14. Решите неравенство: $\tg 3x > \sqrt{3}$

A) $\left[ \cfrac{\pi}{9} + \cfrac{n\pi}{3}; \cfrac{\pi}{6} + \cfrac{n\pi}{3} \right]$
B) $\left( -\cfrac{\pi}{6} + \cfrac{n\pi}{3}; \cfrac{\pi}{6} + \cfrac{n\pi}{3} \right)$
C) $\left[ -\cfrac{\pi}{6} + \cfrac{n\pi}{3}; \cfrac{\pi}{6} + \cfrac{n\pi}{3} \right]$
D) $\left( -\cfrac{\pi}{6} + \cfrac{n\pi}{3}; \cfrac{\pi}{6} + \cfrac{n\pi}{3} \right]$
E) $\left( \cfrac{\pi}{9} + \cfrac{n\pi}{3}; \cfrac{\pi}{6} + \cfrac{n\pi}{3} \right)$

15. Найти угол между касательной к графику функции $f(x)=\cfrac{18}{\sqrt{x}}$ в точке с абсциссой $x_0=3$ и осью $Ox$

A) $-\cfrac{\pi}{4}$
B) $\cfrac{\pi}{3}$
C) $\cfrac{2\pi}{3}$
D) $-\cfrac{2\pi}{3}$
E) $\cfrac{3\pi}{4}$

16. Первообразные функции $f(x)=9(4+3x)^2$ равны: 

A) $2(4+3x)^3$
B) $3(4+3x)^3 + 7$
C) $9(4+3x)^3 + C$
D) $(4+3x)^3 + C$
E) $\cfrac{1}{3}(4+3x)^3$

17. В равнобедренную трапецию вписана окружность с радиусом равным 12 см и боковой стороной равной 25 см. Вычислите площадь этой трапеции. 

A) 680
B) 560
C) 540
D) 600
E) 640

18. Найдите диагонали ромба, зная, что его диагонали относятся как 2:3, а площадь ромба равна 12 с$m^2$

A) 2 см, 3 см.
B) 5 см, 6 см.
C) 3 см, 9 см.
D) 8 см, 12 см.
E) 4 см, 6 см.

19. Проекции двух наклонных равные 10 см и 24 см образуют на плоскости прямой угол. Определите длину перпендикуляра, если наименьшая из наклонных, равна расстоянию между точками пересечения наклонных с плоскостью. 

A) 24 см 
B) 26 см
C) 28 см
D) 30 см
E) 29 см

20. Не вычисляя корней $х_1$ и $х_2$ уравнения $2x^2+5x-3=0$. Найдите $x_1^3+x_2^3$.

A) -26,875
B) 26,875
C) 36
D) -42
E) 25

21. Решите неравенство: $\sin x \cdot \sqrt{9-x^2} > 0$

A) (-3;0)
B) [-3;3]
C) $(-3;0) \cup (0;3)$
D) (-3;3)
E) (0;3)

22. Второй член арифметической прогрессии 9, а еѐ третий член больше первого на 12. Найти $S_{10}$ - ?

A) 280
B) 310
C) 300
D) 290
E) 320

23. Дана функция $f(x)=\ln \cfrac{x-1}{x^2+1}$. Найдите $f'(2)$

A) \cfrac{1}{2}
B) 2
C) \cfrac{1}{3}
D) 5
E) \cfrac{1}{5}

24. В правильной шестиугольной призме $ABCDEFA_1B_1C_1D_1E_1F_1$ диагонали $B_1F$ и $B_1E$ равны соответственно 8см и 10см, тогда площадь основания этой призмы равна:

A) $36\sqrt{3} cm^2$
B) $64\sqrt{3} cm^2$
C) $9\sqrt{3} cm^2$
D) $54\sqrt{3} cm^2$
E) $52\sqrt{3} cm^2$

25. На отрезке прямой $y=-\cfrac{2}{3}x+3$, отсекаемом осями координат, количество точек с целыми координатами равно

A) 1.
B) 2.
C) 3.
D) 0.
E) 4.

\end{document}